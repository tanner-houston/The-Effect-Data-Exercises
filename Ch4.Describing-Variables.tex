% Options for packages loaded elsewhere
\PassOptionsToPackage{unicode}{hyperref}
\PassOptionsToPackage{hyphens}{url}
\PassOptionsToPackage{dvipsnames,svgnames,x11names}{xcolor}
%
\documentclass[
  letterpaper,
  DIV=11,
  numbers=noendperiod]{scrartcl}

\usepackage{amsmath,amssymb}
\usepackage{iftex}
\ifPDFTeX
  \usepackage[T1]{fontenc}
  \usepackage[utf8]{inputenc}
  \usepackage{textcomp} % provide euro and other symbols
\else % if luatex or xetex
  \usepackage{unicode-math}
  \defaultfontfeatures{Scale=MatchLowercase}
  \defaultfontfeatures[\rmfamily]{Ligatures=TeX,Scale=1}
\fi
\usepackage{lmodern}
\ifPDFTeX\else  
    % xetex/luatex font selection
\fi
% Use upquote if available, for straight quotes in verbatim environments
\IfFileExists{upquote.sty}{\usepackage{upquote}}{}
\IfFileExists{microtype.sty}{% use microtype if available
  \usepackage[]{microtype}
  \UseMicrotypeSet[protrusion]{basicmath} % disable protrusion for tt fonts
}{}
\makeatletter
\@ifundefined{KOMAClassName}{% if non-KOMA class
  \IfFileExists{parskip.sty}{%
    \usepackage{parskip}
  }{% else
    \setlength{\parindent}{0pt}
    \setlength{\parskip}{6pt plus 2pt minus 1pt}}
}{% if KOMA class
  \KOMAoptions{parskip=half}}
\makeatother
\usepackage{xcolor}
\setlength{\emergencystretch}{3em} % prevent overfull lines
\setcounter{secnumdepth}{-\maxdimen} % remove section numbering
% Make \paragraph and \subparagraph free-standing
\ifx\paragraph\undefined\else
  \let\oldparagraph\paragraph
  \renewcommand{\paragraph}[1]{\oldparagraph{#1}\mbox{}}
\fi
\ifx\subparagraph\undefined\else
  \let\oldsubparagraph\subparagraph
  \renewcommand{\subparagraph}[1]{\oldsubparagraph{#1}\mbox{}}
\fi

\usepackage{color}
\usepackage{fancyvrb}
\newcommand{\VerbBar}{|}
\newcommand{\VERB}{\Verb[commandchars=\\\{\}]}
\DefineVerbatimEnvironment{Highlighting}{Verbatim}{commandchars=\\\{\}}
% Add ',fontsize=\small' for more characters per line
\usepackage{framed}
\definecolor{shadecolor}{RGB}{241,243,245}
\newenvironment{Shaded}{\begin{snugshade}}{\end{snugshade}}
\newcommand{\AlertTok}[1]{\textcolor[rgb]{0.68,0.00,0.00}{#1}}
\newcommand{\AnnotationTok}[1]{\textcolor[rgb]{0.37,0.37,0.37}{#1}}
\newcommand{\AttributeTok}[1]{\textcolor[rgb]{0.40,0.45,0.13}{#1}}
\newcommand{\BaseNTok}[1]{\textcolor[rgb]{0.68,0.00,0.00}{#1}}
\newcommand{\BuiltInTok}[1]{\textcolor[rgb]{0.00,0.23,0.31}{#1}}
\newcommand{\CharTok}[1]{\textcolor[rgb]{0.13,0.47,0.30}{#1}}
\newcommand{\CommentTok}[1]{\textcolor[rgb]{0.37,0.37,0.37}{#1}}
\newcommand{\CommentVarTok}[1]{\textcolor[rgb]{0.37,0.37,0.37}{\textit{#1}}}
\newcommand{\ConstantTok}[1]{\textcolor[rgb]{0.56,0.35,0.01}{#1}}
\newcommand{\ControlFlowTok}[1]{\textcolor[rgb]{0.00,0.23,0.31}{#1}}
\newcommand{\DataTypeTok}[1]{\textcolor[rgb]{0.68,0.00,0.00}{#1}}
\newcommand{\DecValTok}[1]{\textcolor[rgb]{0.68,0.00,0.00}{#1}}
\newcommand{\DocumentationTok}[1]{\textcolor[rgb]{0.37,0.37,0.37}{\textit{#1}}}
\newcommand{\ErrorTok}[1]{\textcolor[rgb]{0.68,0.00,0.00}{#1}}
\newcommand{\ExtensionTok}[1]{\textcolor[rgb]{0.00,0.23,0.31}{#1}}
\newcommand{\FloatTok}[1]{\textcolor[rgb]{0.68,0.00,0.00}{#1}}
\newcommand{\FunctionTok}[1]{\textcolor[rgb]{0.28,0.35,0.67}{#1}}
\newcommand{\ImportTok}[1]{\textcolor[rgb]{0.00,0.46,0.62}{#1}}
\newcommand{\InformationTok}[1]{\textcolor[rgb]{0.37,0.37,0.37}{#1}}
\newcommand{\KeywordTok}[1]{\textcolor[rgb]{0.00,0.23,0.31}{#1}}
\newcommand{\NormalTok}[1]{\textcolor[rgb]{0.00,0.23,0.31}{#1}}
\newcommand{\OperatorTok}[1]{\textcolor[rgb]{0.37,0.37,0.37}{#1}}
\newcommand{\OtherTok}[1]{\textcolor[rgb]{0.00,0.23,0.31}{#1}}
\newcommand{\PreprocessorTok}[1]{\textcolor[rgb]{0.68,0.00,0.00}{#1}}
\newcommand{\RegionMarkerTok}[1]{\textcolor[rgb]{0.00,0.23,0.31}{#1}}
\newcommand{\SpecialCharTok}[1]{\textcolor[rgb]{0.37,0.37,0.37}{#1}}
\newcommand{\SpecialStringTok}[1]{\textcolor[rgb]{0.13,0.47,0.30}{#1}}
\newcommand{\StringTok}[1]{\textcolor[rgb]{0.13,0.47,0.30}{#1}}
\newcommand{\VariableTok}[1]{\textcolor[rgb]{0.07,0.07,0.07}{#1}}
\newcommand{\VerbatimStringTok}[1]{\textcolor[rgb]{0.13,0.47,0.30}{#1}}
\newcommand{\WarningTok}[1]{\textcolor[rgb]{0.37,0.37,0.37}{\textit{#1}}}

\providecommand{\tightlist}{%
  \setlength{\itemsep}{0pt}\setlength{\parskip}{0pt}}\usepackage{longtable,booktabs,array}
\usepackage{calc} % for calculating minipage widths
% Correct order of tables after \paragraph or \subparagraph
\usepackage{etoolbox}
\makeatletter
\patchcmd\longtable{\par}{\if@noskipsec\mbox{}\fi\par}{}{}
\makeatother
% Allow footnotes in longtable head/foot
\IfFileExists{footnotehyper.sty}{\usepackage{footnotehyper}}{\usepackage{footnote}}
\makesavenoteenv{longtable}
\usepackage{graphicx}
\makeatletter
\def\maxwidth{\ifdim\Gin@nat@width>\linewidth\linewidth\else\Gin@nat@width\fi}
\def\maxheight{\ifdim\Gin@nat@height>\textheight\textheight\else\Gin@nat@height\fi}
\makeatother
% Scale images if necessary, so that they will not overflow the page
% margins by default, and it is still possible to overwrite the defaults
% using explicit options in \includegraphics[width, height, ...]{}
\setkeys{Gin}{width=\maxwidth,height=\maxheight,keepaspectratio}
% Set default figure placement to htbp
\makeatletter
\def\fps@figure{htbp}
\makeatother

\usepackage{float}
\usepackage{tabularray}
\usepackage[normalem]{ulem}
\usepackage{graphicx}
\UseTblrLibrary{booktabs}
\UseTblrLibrary{siunitx}
\NewTableCommand{\tinytableDefineColor}[3]{\definecolor{#1}{#2}{#3}}
\newcommand{\tinytableTabularrayUnderline}[1]{\underline{#1}}
\newcommand{\tinytableTabularrayStrikeout}[1]{\sout{#1}}
\KOMAoption{captions}{tableheading}
\makeatletter
\@ifpackageloaded{caption}{}{\usepackage{caption}}
\AtBeginDocument{%
\ifdefined\contentsname
  \renewcommand*\contentsname{Table of contents}
\else
  \newcommand\contentsname{Table of contents}
\fi
\ifdefined\listfigurename
  \renewcommand*\listfigurename{List of Figures}
\else
  \newcommand\listfigurename{List of Figures}
\fi
\ifdefined\listtablename
  \renewcommand*\listtablename{List of Tables}
\else
  \newcommand\listtablename{List of Tables}
\fi
\ifdefined\figurename
  \renewcommand*\figurename{Figure}
\else
  \newcommand\figurename{Figure}
\fi
\ifdefined\tablename
  \renewcommand*\tablename{Table}
\else
  \newcommand\tablename{Table}
\fi
}
\@ifpackageloaded{float}{}{\usepackage{float}}
\floatstyle{ruled}
\@ifundefined{c@chapter}{\newfloat{codelisting}{h}{lop}}{\newfloat{codelisting}{h}{lop}[chapter]}
\floatname{codelisting}{Listing}
\newcommand*\listoflistings{\listof{codelisting}{List of Listings}}
\makeatother
\makeatletter
\makeatother
\makeatletter
\@ifpackageloaded{caption}{}{\usepackage{caption}}
\@ifpackageloaded{subcaption}{}{\usepackage{subcaption}}
\makeatother
\ifLuaTeX
  \usepackage{selnolig}  % disable illegal ligatures
\fi
\usepackage{bookmark}

\IfFileExists{xurl.sty}{\usepackage{xurl}}{} % add URL line breaks if available
\urlstyle{same} % disable monospaced font for URLs
\hypersetup{
  pdftitle={The Effect: Describing Variables},
  pdfauthor={Tanner Houston},
  colorlinks=true,
  linkcolor={blue},
  filecolor={Maroon},
  citecolor={Blue},
  urlcolor={Blue},
  pdfcreator={LaTeX via pandoc}}

\title{The Effect: Describing Variables}
\author{Tanner Houston}
\date{}

\begin{document}
\maketitle

\subsection{Chapter 4: Describing
Variables}\label{chapter-4-describing-variables}

\begin{Shaded}
\begin{Highlighting}[]
\FunctionTok{library}\NormalTok{(tidyverse)}
\end{Highlighting}
\end{Shaded}

\begin{verbatim}
-- Attaching core tidyverse packages ------------------------ tidyverse 2.0.0 --
v dplyr     1.1.4     v readr     2.1.5
v forcats   1.0.0     v stringr   1.5.1
v ggplot2   3.5.1     v tibble    3.2.1
v lubridate 1.9.3     v tidyr     1.3.1
v purrr     1.0.2     
-- Conflicts ------------------------------------------ tidyverse_conflicts() --
x dplyr::filter() masks stats::filter()
x dplyr::lag()    masks stats::lag()
i Use the conflicted package (<http://conflicted.r-lib.org/>) to force all conflicts to become errors
\end{verbatim}

\begin{Shaded}
\begin{Highlighting}[]
\FunctionTok{library}\NormalTok{(modelsummary)}
\end{Highlighting}
\end{Shaded}

\begin{verbatim}
`modelsummary` 2.0.0 now uses `tinytable` as its default table-drawing
  backend. Learn more at: https://vincentarelbundock.github.io/tinytable/

Revert to `kableExtra` for one session:

  options(modelsummary_factory_default = 'kableExtra')
  options(modelsummary_factory_latex = 'kableExtra')
  options(modelsummary_factory_html = 'kableExtra')

Silence this message forever:

  config_modelsummary(startup_message = FALSE)
\end{verbatim}

\begin{Shaded}
\begin{Highlighting}[]
\FunctionTok{library}\NormalTok{(causaldata)}
\CommentTok{\#??causaldata}
\FunctionTok{view}\NormalTok{(causaldata}\SpecialCharTok{::}\NormalTok{Mroz)}


\NormalTok{df }\OtherTok{\textless{}{-}}\NormalTok{ causaldata}\SpecialCharTok{::}\NormalTok{Mroz }\SpecialCharTok{|\textgreater{}}
  \FunctionTok{filter}\NormalTok{(lfp }\SpecialCharTok{==} \ConstantTok{TRUE}\NormalTok{) }\SpecialCharTok{|\textgreater{}}
  \FunctionTok{mutate}\NormalTok{(}\AttributeTok{earn =} \FunctionTok{exp}\NormalTok{(lwg))}
\end{Highlighting}
\end{Shaded}

\paragraph{1. Draw a scatterplot}\label{draw-a-scatterplot}

\begin{Shaded}
\begin{Highlighting}[]
\FunctionTok{ggplot}\NormalTok{(df, }\FunctionTok{aes}\NormalTok{(}\AttributeTok{x =}\NormalTok{ inc, }\AttributeTok{y =}\NormalTok{ earn)) }\SpecialCharTok{+}
  \FunctionTok{geom\_point}\NormalTok{() }\SpecialCharTok{+}
  \CommentTok{\# use a log scale for the axes}
  \FunctionTok{scale\_x\_log10}\NormalTok{() }\SpecialCharTok{+}
  \FunctionTok{scale\_y\_log10}\NormalTok{()}
\end{Highlighting}
\end{Shaded}

\begin{verbatim}
Warning in transformation$transform(x): NaNs produced
\end{verbatim}

\begin{verbatim}
Warning in scale_x_log10(): log-10 transformation introduced infinite values.
\end{verbatim}

\begin{verbatim}
Warning: Removed 1 row containing missing values or values outside the scale range
(`geom_point()`).
\end{verbatim}

\includegraphics{Ch4.Describing-Variables_files/figure-pdf/unnamed-chunk-2-1.pdf}

\paragraph{2. Conditional Mean by College
Attendance}\label{conditional-mean-by-college-attendance}

\begin{Shaded}
\begin{Highlighting}[]
\NormalTok{df }\SpecialCharTok{|\textgreater{}}
  \FunctionTok{group\_by}\NormalTok{(wc) }\SpecialCharTok{|\textgreater{}}
  \FunctionTok{summarize}\NormalTok{(}\AttributeTok{earn =} \FunctionTok{mean}\NormalTok{(earn))}
\end{Highlighting}
\end{Shaded}

\begin{verbatim}
# A tibble: 2 x 2
  wc     earn
  <lgl> <dbl>
1 FALSE  3.58
2 TRUE   5.35
\end{verbatim}

\paragraph{3. Conditional Mean by Bins}\label{conditional-mean-by-bins}

\begin{Shaded}
\begin{Highlighting}[]
\NormalTok{df }\SpecialCharTok{|\textgreater{}}
  \CommentTok{\# cut() cuts the variable into bins}
  \FunctionTok{mutate}\NormalTok{(}\AttributeTok{inc\_cut =} \FunctionTok{cut}\NormalTok{(inc, }\DecValTok{10}\NormalTok{)) }\SpecialCharTok{|\textgreater{}}
  \FunctionTok{group\_by}\NormalTok{(inc\_cut) }\SpecialCharTok{|\textgreater{}}
  \FunctionTok{summarize}\NormalTok{(}\AttributeTok{earn =} \FunctionTok{mean}\NormalTok{(earn),}
            \AttributeTok{count =} \FunctionTok{n}\NormalTok{())}
\end{Highlighting}
\end{Shaded}

\begin{verbatim}
# A tibble: 10 x 3
   inc_cut       earn count
   <fct>        <dbl> <int>
 1 (-0.12,9.07]  3.09    47
 2 (9.07,18.2]   3.83   200
 3 (18.2,27.3]   4.95   116
 4 (27.3,36.4]   4.56    40
 5 (36.4,45.5]   4.24    15
 6 (45.5,54.6]   6.14     6
 7 (54.6,63.7]   3.83     1
 8 (63.7,72.8]   5.23     1
 9 (72.8,81.9]   7.02     1
10 (81.9,91.1]   3.67     1
\end{verbatim}

\paragraph{4. Draw the LOESS and Linear Regression
Curves}\label{draw-the-loess-and-linear-regression-curves}

\begin{Shaded}
\begin{Highlighting}[]
\CommentTok{\#LOESS}
\FunctionTok{ggplot}\NormalTok{(df, }\FunctionTok{aes}\NormalTok{(}\AttributeTok{x =}\NormalTok{ inc, }\AttributeTok{y =}\NormalTok{ earn)) }\SpecialCharTok{+}
  \FunctionTok{geom\_point}\NormalTok{() }\SpecialCharTok{+}
  \CommentTok{\# geom smooth by default draws a LOESS}
  \FunctionTok{geom\_smooth}\NormalTok{(}\AttributeTok{se =} \ConstantTok{FALSE}\NormalTok{) }\SpecialCharTok{+}
  \FunctionTok{scale\_x\_log10}\NormalTok{() }\SpecialCharTok{+} \FunctionTok{scale\_y\_log10}\NormalTok{()}
\end{Highlighting}
\end{Shaded}

\begin{verbatim}
Warning in transformation$transform(x): NaNs produced
\end{verbatim}

\begin{verbatim}
Warning in scale_x_log10(): log-10 transformation introduced infinite values.
\end{verbatim}

\begin{verbatim}
Warning in transformation$transform(x): NaNs produced
\end{verbatim}

\begin{verbatim}
Warning in scale_x_log10(): log-10 transformation introduced infinite values.
\end{verbatim}

\begin{verbatim}
`geom_smooth()` using method = 'loess' and formula = 'y ~ x'
\end{verbatim}

\begin{verbatim}
Warning: Removed 1 row containing non-finite outside the scale range
(`stat_smooth()`).
\end{verbatim}

\begin{verbatim}
Warning: Removed 1 row containing missing values or values outside the scale range
(`geom_point()`).
\end{verbatim}

\includegraphics{Ch4.Describing-Variables_files/figure-pdf/unnamed-chunk-5-1.pdf}

\begin{Shaded}
\begin{Highlighting}[]
\CommentTok{\#Linear regression method}
\FunctionTok{ggplot}\NormalTok{(df , }\FunctionTok{aes}\NormalTok{(}\AttributeTok{x =}\NormalTok{ inc, }\AttributeTok{y =}\NormalTok{ earn)) }\SpecialCharTok{+}
  \FunctionTok{geom\_point}\NormalTok{() }\SpecialCharTok{+}
  \FunctionTok{geom\_smooth}\NormalTok{(}\AttributeTok{method =} \StringTok{"lm"}\NormalTok{, }\AttributeTok{se =} \ConstantTok{FALSE}\NormalTok{) }\SpecialCharTok{+}
  \FunctionTok{scale\_x\_log10}\NormalTok{() }\SpecialCharTok{+}
  \FunctionTok{scale\_y\_log10}\NormalTok{()}
\end{Highlighting}
\end{Shaded}

\begin{verbatim}
Warning in transformation$transform(x): NaNs produced
\end{verbatim}

\begin{verbatim}
Warning in scale_x_log10(): log-10 transformation introduced infinite values.
\end{verbatim}

\begin{verbatim}
Warning in transformation$transform(x): NaNs produced
\end{verbatim}

\begin{verbatim}
Warning in scale_x_log10(): log-10 transformation introduced infinite values.
\end{verbatim}

\begin{verbatim}
`geom_smooth()` using formula = 'y ~ x'
\end{verbatim}

\begin{verbatim}
Warning: Removed 1 row containing non-finite outside the scale range
(`stat_smooth()`).
\end{verbatim}

\begin{verbatim}
Warning: Removed 1 row containing missing values or values outside the scale range
(`geom_point()`).
\end{verbatim}

\includegraphics{Ch4.Describing-Variables_files/figure-pdf/unnamed-chunk-5-2.pdf}

\paragraph{5. Run a Linear Regression}\label{run-a-linear-regression}

\begin{Shaded}
\begin{Highlighting}[]
\NormalTok{modl1 }\OtherTok{\textless{}{-}} \FunctionTok{lm}\NormalTok{(lwg}\SpecialCharTok{\textasciitilde{}} \FunctionTok{log}\NormalTok{(inc), }\AttributeTok{data =}\NormalTok{ df)}
\end{Highlighting}
\end{Shaded}

\begin{verbatim}
Warning in log(inc): NaNs produced
\end{verbatim}

\begin{Shaded}
\begin{Highlighting}[]
\CommentTok{\#with controls}
\NormalTok{modl2 }\OtherTok{\textless{}{-}} \FunctionTok{lm}\NormalTok{(lwg}\SpecialCharTok{\textasciitilde{}} \FunctionTok{log}\NormalTok{(inc) }\SpecialCharTok{+}\NormalTok{ wc }\SpecialCharTok{+}\NormalTok{ k618, }\AttributeTok{data =}\NormalTok{ df)}
\end{Highlighting}
\end{Shaded}

\begin{verbatim}
Warning in log(inc): NaNs produced
\end{verbatim}

\begin{Shaded}
\begin{Highlighting}[]
\CommentTok{\# make a table with the models}
\FunctionTok{msummary}\NormalTok{(}\FunctionTok{list}\NormalTok{(modl1, modl2))}
\end{Highlighting}
\end{Shaded}

\begin{table}
\centering
\begin{tblr}[         %% tabularray outer open
]                     %% tabularray outer close
{                     %% tabularray inner open
colspec={Q[]Q[]Q[]},
column{1}={halign=l,},
column{2}={halign=c,},
column{3}={halign=c,},
hline{10}={1,2,3}{solid, 0.05em, black},
}                     %% tabularray inner close
\toprule
& (1) & (2) \\ \midrule %% TinyTableHeader
(Intercept) & \num{0.569}    & \num{0.742}    \\
& (\num{0.185})  & (\num{0.183})  \\
log(inc)    & \num{0.221}    & \num{0.151}    \\
& (\num{0.065})  & (\num{0.066})  \\
wcTRUE      &                 & \num{0.310}    \\
&                 & (\num{0.075})  \\
k618        &                 & \num{-0.061}   \\
&                 & (\num{0.026})  \\
Num.Obs.    & \num{427}      & \num{427}      \\
R2          & \num{0.027}    & \num{0.081}    \\
R2 Adj.     & \num{0.024}    & \num{0.075}    \\
AIC         & \num{928.7}    & \num{908.1}    \\
BIC         & \num{940.9}    & \num{928.4}    \\
Log.Lik.    & \num{-461.340} & \num{-449.050} \\
RMSE        & \num{0.71}     & \num{0.69}     \\
\bottomrule
\end{tblr}
\end{table}



\end{document}
